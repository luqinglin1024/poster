\documentclass[border=2pt]{standalone}
\usepackage{tikz}


% This tex, loads the Breast GT pallete if is not defined.
% The document takes advantage of the xcolor package primitive 
% \def\@ifundefinedcolor#1{\@ifundefined{\string\color@#1}}
% therefore it xcolor package is needed or the definitions needs to be added.
% 
% TODO: 
% 	create a more generic script that checks if all the packages are there otherwise loads them.
%	or defines the missing primitive.
%   take a look at: \@ifpackageloaded{<name>}{<true>}{<false>}
% 					http://tex.stackexchange.com/questions/16199/test-if-a-package-or-package-option-is-loaded

\makeatletter
\newcommand{\colorprovide}[2]{%
  \@ifundefinedcolor{#1}{\colorlet{#1}{#2}}{}}

\newcommand{\defineColorWhenNoExist}[3]{%
  \@ifundefinedcolor{#1}{\definecolor{#1}{#2}{#3}}{}}
\makeatother

\defineColorWhenNoExist{bgColor}{rgb}{0.0000, 0.0000, 0.0000}
\defineColorWhenNoExist{boundaryColor}{rgb}{0.8784, 0.8784, 0.7529}
\defineColorWhenNoExist{chestWallColor}{rgb}{0.5294, 0.7843, 0.6078}
\defineColorWhenNoExist{fatColor}{rgb}{0.9804, 0.5882, 0.1176}
\defineColorWhenNoExist{fibroGlandColor}{rgb}{1.0000, 1.0000, 0.0000}
\defineColorWhenNoExist{lesionColor}{rgb}{1.0000, 0.2510, 0.0000}
\defineColorWhenNoExist{lungColor}{rgb}{0.2353, 0.6078, 0.8235}
\defineColorWhenNoExist{pectoralColor}{rgb}{0.6510, 0.3490, 1.0000}
\defineColorWhenNoExist{ribColor}{rgb}{0.0000, 0.4510, 0.1961}
\defineColorWhenNoExist{skinColor}{rgb}{0.9804, 0.7255, 0.7451}
\defineColorWhenNoExist{unkTissueColor}{rgb}{0.6000, 0.3020, 0.2510}


\begin{document}
\begin{tikzpicture}
%\node[anchor=south east,inner sep=0] (mappedGTNode) at (0,0) {\includegraphics[trim=115 250 118 250,clip,height=.12\textwidth]{mapGTintoSIFT}};

\node[anchor=south west,inner sep=0](occurrenceMap) at (-8pt,0) {\includegraphics[height=.12\textwidth]{siftOccurrences2}};


\tikzset{labelSt/.style=
{anchor=north west,rectangle,
node distance=1.96pt,
scale=.6,
minimum width=1pt,minimum height=1pt,
}}

\node[anchor=south west,inner sep=0, right=of occurrenceMap](imgNode) {\includegraphics[trim = 5 5 18 0, clip,width=.12\textwidth]{classPrior}};

\draw[] (imgNode.south west) +(14pt,3.5pt) node[labelSt, draw=bgColor, fill=bgColor] (bgName) {} ;
\draw[] node[labelSt,draw=lungColor, fill=lungColor,right=of bgName] (lungName) {};
\draw[] node[labelSt, draw=chestWallColor,right=of lungName, fill=chestWallColor] (cwName) {} ;
\draw[] node[labelSt, draw=ribColor, fill=ribColor,right=of cwName] (ribName) {} ;
\draw[] node[labelSt, draw=pectoralColor, fill=pectoralColor,right=of ribName] (pectoralName) {} ;
\draw[] node[labelSt, draw=fibroGlandColor, fill=fibroGlandColor,right=of pectoralName] (fibroGlandName) {} ;
\draw[] node[labelSt, draw=fatColor, fill=fatColor,right=of fibroGlandName] (fatName) {} ;
\draw[] node[labelSt, draw=skinColor, fill=skinColor,right=of fatName] (skinName) {} ;
\draw[] node[labelSt, draw=lesionColor, fill=lesionColor,right=of skinName] (lesionName) {} ;
\draw[] node[labelSt, draw=boundaryColor, fill=boundaryColor,right=of lesionName] (boundaryName) {} ;
\end{tikzpicture}
\end{document}
