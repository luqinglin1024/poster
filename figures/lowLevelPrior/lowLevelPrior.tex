\documentclass[border=2pt]{standalone}
\usepackage{tikz}


\input{../../defineBreastGTlabelColors.tex}

\begin{document}
\begin{tikzpicture}

\newcommand\mySize{6cm} 
  
%\node[anchor=south east,inner sep=0] (mappedGTNode) at (0,0) {\includegraphics[trim=115 250 118 250,clip,height=\mySize]{mapGTintoSIFT}};

\node[anchor=south west,inner sep=0](occurrenceMap) at (-8pt,0) {\includegraphics[height=\mySize]{siftOccurrences2}};


\tikzset{labelSt/.style=
{anchor=north west,rectangle,
node distance=1.96pt,
scale=.6,
minimum width=1pt,minimum height=1pt,
}}

\node[anchor=south west,inner sep=0, above=of occurrenceMap](imgNode) {\includegraphics[trim = 5 5 18 0, clip,width=\mySize]{classPrior}};

\draw[] (imgNode.south west) +(14pt,3.5pt) node[labelSt, draw=bgColor, fill=bgColor] (bgName) {} ;
\draw[] node[labelSt,draw=lungColor, fill=lungColor,right=of bgName] (lungName) {};
\draw[] node[labelSt, draw=chestWallColor,right=of lungName, fill=chestWallColor] (cwName) {} ;
\draw[] node[labelSt, draw=ribColor, fill=ribColor,right=of cwName] (ribName) {} ;
\draw[] node[labelSt, draw=pectoralColor, fill=pectoralColor,right=of ribName] (pectoralName) {} ;
\draw[] node[labelSt, draw=fibroGlandColor, fill=fibroGlandColor,right=of pectoralName] (fibroGlandName) {} ;
\draw[] node[labelSt, draw=fatColor, fill=fatColor,right=of fibroGlandName] (fatName) {} ;
\draw[] node[labelSt, draw=skinColor, fill=skinColor,right=of fatName] (skinName) {} ;
\draw[] node[labelSt, draw=lesionColor, fill=lesionColor,right=of skinName] (lesionName) {} ;
\draw[] node[labelSt, draw=boundaryColor, fill=boundaryColor,right=of lesionName] (boundaryName) {} ;
\end{tikzpicture}
\end{document}
