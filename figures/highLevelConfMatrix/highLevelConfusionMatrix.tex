\documentclass[border=2pt]{standalone}
\usepackage{tikz}
\usepackage{pgfplots}
\usepackage{scalefnt}

% This tex, loads the Breast GT pallete if is not defined.
% The document takes advantage of the xcolor package primitive 
% \def\@ifundefinedcolor#1{\@ifundefined{\string\color@#1}}
% therefore it xcolor package is needed or the definitions needs to be added.
% 
% TODO: 
% 	create a more generic script that checks if all the packages are there otherwise loads them.
%	or defines the missing primitive.
%   take a look at: \@ifpackageloaded{<name>}{<true>}{<false>}
% 					http://tex.stackexchange.com/questions/16199/test-if-a-package-or-package-option-is-loaded

\makeatletter
\newcommand{\colorprovide}[2]{%
  \@ifundefinedcolor{#1}{\colorlet{#1}{#2}}{}}

\newcommand{\defineColorWhenNoExist}[3]{%
  \@ifundefinedcolor{#1}{\definecolor{#1}{#2}{#3}}{}}
\makeatother

\defineColorWhenNoExist{bgColor}{rgb}{0.0000, 0.0000, 0.0000}
\defineColorWhenNoExist{boundaryColor}{rgb}{0.8784, 0.8784, 0.7529}
\defineColorWhenNoExist{chestWallColor}{rgb}{0.5294, 0.7843, 0.6078}
\defineColorWhenNoExist{fatColor}{rgb}{0.9804, 0.5882, 0.1176}
\defineColorWhenNoExist{fibroGlandColor}{rgb}{1.0000, 1.0000, 0.0000}
\defineColorWhenNoExist{lesionColor}{rgb}{1.0000, 0.2510, 0.0000}
\defineColorWhenNoExist{lungColor}{rgb}{0.2353, 0.6078, 0.8235}
\defineColorWhenNoExist{pectoralColor}{rgb}{0.6510, 0.3490, 1.0000}
\defineColorWhenNoExist{ribColor}{rgb}{0.0000, 0.4510, 0.1961}
\defineColorWhenNoExist{skinColor}{rgb}{0.9804, 0.7255, 0.7451}
\defineColorWhenNoExist{unkTissueColor}{rgb}{0.6000, 0.3020, 0.2510}

\begin{document}
\makeatletter
  \newcommand\tinyv{\@setfontsize\tinyv{7pt}{6}}
\makeatother

\newcommand\confMatrixSize{9cm}
\newcommand\gtLabelDistance{0.3cm}

\pgfplotsset{
	myStyleCommon/.append style={	width=\confMatrixSize,
                                height=\confMatrixSize,
                                scale only axis,
                                xlabel={\small Estimated Class},
                                ylabel={\small Actual Class},
                                yticklabels={},
                                xticklabels={},
                                xticklabel pos=right,
                                xlabel near ticks,
                                xtick={1.5,2.5,...,7.5},
                                ytick={1.5,2.5,...,7.5},
                                grid,
                              } }

\pgfplotsset{
	confMatrix/.append	style={	myStyleCommon,
                              axis on top,
                              colormap/jet,
                              point meta min=0,
                              point meta max=1,
                              xmin=0.5,
                              xmax=8.5,
                              y dir=reverse,
                              ymin=0.5,
                              ymax=8.5,
							              } }

\pgfplotsset{							  
	gtLabelsAxis/.append style= { myStyleCommon,
                                axis lines=none,
                                colormap={labelsGT}{
                                          color(0cm)=(lungColor);	
                                          color(1cm)=(chestWallColor);
                                          color(2cm)=(ribColor);
                                          color(3cm)=(pectoralColor);
                                          color(4cm)=(fibroGlandColor); 
                                          color(5cm)=(fatColor); 
                                          color(6cm)=(skinColor);    
                                          color(7cm)=(lesionColor);
                                          },
                                  point meta min=0,
                                  point meta max=7,
                                  xmin=-0.5,xmax=7.5,
                                  height=0cm,
							                } }

\tikzstyle{gtLabelsPlotStyle}= [scatter,
                                only marks,
                                mark=square*,
                                mark size=5pt,
                                domain=0:7,
                                samples at={0,...,7},
                               ]

\tikzstyle{stdDrawingStyle} = [scatter, only marks, no marks,]
\tikzstyle{meanValueStyle}  = [ font=\tinyv,
                                fill=white,
                                inner sep = 2pt,
                                fill opacity=0.5,
                                text opacity=1,
                                %anchor=\coordindex,
                              ]
\tikzstyle{stdValuesStyle}  = [ meanValueStyle,
                                anchor=north,
                              ]
 

\begin{tikzpicture}
\begin{axis}
  [ confMatrix,
    %colorbar,colorbar to name={storedcolorbar},
    name=siftConfMatrix,
    %title=SIFT,
  ]
  \addplot graphics [xmin=0.5,xmax=8.5,ymin=0.5,ymax=8.5] 
  {/home/sik/Work/escola/recerca/iwdm2014/poster/figures/highLevelConfMatrix/highLevelSIFTConfusionMatrixMean.png};

  \addplot
    [ stdDrawingStyle,
      % we use ’point meta’ as color data...
      point meta=\thisrow{siftStd},
      % ... therefore, we can’t use it as argument for nodes near coords ..
      % ... look how to print the numbers at pgfplotstable.pdf manual
      nodes near coords*=
        {$
        \pgfmathprintnumber[fixed,precision=0]{\meanVal}\%
        $},
      % ... which requires to define a visualization dependency:
      visualization depends on={100*\thisrow{siftMean} \as \meanVal},
      % Add the node visual cues
      every node near coord/.append style= {meanValueStyle },
    ]
  table[ y=actualClass, x=estimatedClass ]
  {/home/sik/Work/escola/recerca/iwdm2014/poster/figures/highLevelConfMatrix/highLevelFeaturesConfusion.dat};

  \addplot % Like the previous plot, but this time with less unncecessary stuff.
    [ stdDrawingStyle,
      nodes near coords*=
        {$
        \pm\hspace{-0.1em}
        \pgfmathprintnumber[fixed,precision=1]{\stdVal}
        $},
      visualization depends on={100*\thisrow{siftStd} \as \stdVal},
      every node near coord/.append style= {stdValuesStyle},
    ]
  table[ y=actualClass, x=estimatedClass ]
  {/home/sik/Work/escola/recerca/iwdm2014/poster/figures/highLevelConfMatrix/highLevelFeaturesConfusion.dat};

\end{axis}

\begin{axis}[gtLabelsAxis,
			 at=(siftConfMatrix.north),anchor=south,
			 yshift=\gtLabelDistance,
			 name=xGTLabelsSIFT,
			 ]
			\addplot [gtLabelsPlotStyle,point meta=x] {0};
\end{axis}
 
\begin{axis}[	gtLabelsAxis,
				rotate=-90,yshift=-\gtLabelDistance,
			 	name=yGTLabelsSIFT,
			 ]
			\addplot [gtLabelsPlotStyle,point meta=x] {0};   	
\end{axis}
\node at (siftConfMatrix.south) 
      [ anchor=north,
        yshift=-15pt,
      ]{SIFT};
\end{tikzpicture}
%intensityMean	intensityStd
%
\begin{tikzpicture}
\begin{axis}
  [ confMatrix,
    %colorbar,colorbar to name={storedcolorbar},
    name=intensityConfMatrix,
    %title=SIFT,
  ]
  \addplot graphics [xmin=0.5,xmax=8.5,ymin=0.5,ymax=8.5] 
  {/home/sik/Work/escola/recerca/iwdm2014/poster/figures/highLevelConfMatrix/highLevelIntensityConfusionMatrixMean.png};

  \addplot
    [ stdDrawingStyle,
      % we use ’point meta’ as color data...
      point meta=\thisrow{siftStd},
      % ... therefore, we can’t use it as argument for nodes near coords ..
      % ... look how to print the numbers at pgfplotstable.pdf manual
      nodes near coords*=
        {$
        \pgfmathprintnumber[fixed,precision=0]{\meanVal}\%
        $},
      % ... which requires to define a visualization dependency:
      visualization depends on={100*\thisrow{intensityMean} \as \meanVal},
      % Add the node visual cues
      every node near coord/.append style= {meanValueStyle },
    ]
  table[ y=actualClass, x=estimatedClass ]
  {/home/sik/Work/escola/recerca/iwdm2014/poster/figures/highLevelConfMatrix/highLevelFeaturesConfusion.dat};

  \addplot % Like the previous plot, but this time with less unncecessary stuff.
    [ stdDrawingStyle,
      nodes near coords*=
        {$
        \pm\hspace{-0.1em}
        \pgfmathprintnumber[fixed,precision=1]{\stdVal}
        $},
      visualization depends on={100*\thisrow{intensityStd} \as \stdVal},
      every node near coord/.append style= {stdValuesStyle},
    ]
  table[ y=actualClass, x=estimatedClass ]
  {/home/sik/Work/escola/recerca/iwdm2014/poster/figures/highLevelConfMatrix/highLevelFeaturesConfusion.dat};

\end{axis}

\begin{axis}[gtLabelsAxis,
			 at=(siftConfMatrix.north),anchor=south,
			 yshift=\gtLabelDistance,
			 name=xGTLabelsIntensity,
			 ]
			\addplot [gtLabelsPlotStyle,point meta=x] {0};
\end{axis}
 
\begin{axis}[	gtLabelsAxis,
				rotate=-90,yshift=-\gtLabelDistance,
			 	name=yGTLabelsIntensity,
			 ]
			\addplot [gtLabelsPlotStyle,point meta=x] {0};   	
\end{axis}
\node at (intensityConfMatrix.south) 
      [ anchor=north,
        yshift=-10pt,
      ]{Intensity};
\end{tikzpicture}
\end{document}
