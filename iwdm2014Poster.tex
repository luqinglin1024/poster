\documentclass[25pt, a0paper, portrait, margin=0mm, innermargin=15mm, blockverticalspace=15mm, colspace=15mm, subcolspace=8mm]{tikzposter}

\usepackage{amsmath}
\usepackage{epsf,graphicx,subfig}
\setcounter{tocdepth}{3}
\usepackage{color}
\usepackage{lineno}
\usepackage[nolist]{acronym}
\usepackage{epsf,graphicx,subfig}
\usepackage{amssymb,amsmath}
\usepackage{tikz}
\usetikzlibrary{positioning}
\usepackage{todonotes}
\usepackage{standalone}
\usepackage{scalefnt}
\usepackage{url}

\definecolor{udgColor}{RGB}{82,119,213}


\title{\acs{sift} texture description for understanding breast ultrasound images}
\author{Joan Massich, Fabrice Meriaudeau, Melcior Sent{\'i}s, Sergi Ganau, Elsa~P{\'e}rez, Domenec Puig, Robert  Mart{\'i}, Arnau Oliver and Joan Mart{\'i}}

%\institute{contact author: sik@eia.udg.edu}
%\usetheme{Autumn}\usecolorstyle[colorPalette=BrownBlueOrange]{Germany}
\usetheme{Simple}
%\usecolorstyle[colorOne=udgColor]{Russia} 
\usecolorstyle[colorOne=udgColor]{Denmark} 

\begin{document}\maketitle
/home/sik/Work/escola/recerca/iwdm2014/paper/acronyms.tex

\graphicspath{{figures/paperFigures/}}
\acresetall

\block{Abstract}{
Texture is a powerful cue for describing structures that show a high degree of similarity in their image intensity patterns. This paper describes the use of \acf{sift}, both as low-level and high-level descriptors, applied to differentiate the tissues present in breast US images. For the low-level texture descriptors case, \ac{sift} descriptors are extracted from a regular grid. The high-level texture descriptor is build as a \ac{bof} of \ac{sift} descriptors. 
Experimental results are provided showing the validity of the proposed approach for describing the tissues in breast US images.
}
%\begin{keywords}
%breast cancer, ultrasound, texture, SIFT
%\end{keywords}

\begin{columns} \column{0.45}

\block{Problem definition}{
xxxxxxxxxxxxxxxxxxxxxxxxxx
 
}





\end{columns}
\block[titleoffsety=-1cm,bodyoffsety=-1cm]{Sample document}{This poster...}
%\note[targetoffsetx=24cm, targetoffsety=-9cm,radius=8cm,width=.75\textwidth,innersep=.4cm]{You can...}
\note[targetoffsetx=0cm,targetoffsety=-9cm,radius=8cm,width=.5\textwidth,innersep=.4cm]{You can...}
\end{document}