\documentclass[25pt, a0paper, portrait, margin=0mm, innermargin=15mm, blockverticalspace=10mm, colspace=10mm, subcolspace=8mm]{tikzposter}

\usepackage{amsmath}
\usepackage{epsf,graphicx,subfig}
\setcounter{tocdepth}{3}
\usepackage{xcolor}
\usepackage{lineno}
\usepackage[nolist]{acronym}
\usepackage{epsf,graphicx,subfig}
\usepackage{amssymb,amsmath}
\usepackage{tikz}
\usetikzlibrary{positioning}
\usepackage{todonotes}
\usepackage{standalone}
\usepackage{scalefnt}
\usepackage{url}

% include pgfplots and do a workaround to add the missing layers. This problem might be related to pgfplots and tikz versions as pointed out in: http://tex.stackexchange.com/questions/104010/why-does-loading-pgfplots-after-tikz-break-the-default-layers-in-a-tikzpicture

\usepackage{pgfplots}

% define layers
    \pgfdeclarelayer{foreground}
    \pgfdeclarelayer{background}
    \pgfdeclarelayer{backgroundlayer} 
    \pgfdeclarelayer{notelayer}
% tell TikZ how to stack them (back to front)
    \pgfsetlayers{background,backgroundlayer,main,notelayer,foreground}

\definecolor{udgColor}{RGB}{82,119,213}

\title{\acs{sift} texture description for understanding breast ultrasound images}
\author{Joan Massich, Fabrice Meriaudeau, Melcior Sent{\'i}s, Sergi Ganau, Elsa~P{\'e}rez, Domenec Puig, Robert  Mart{\'i}, Arnau Oliver and Joan Mart{\'i}}

%\institute{contact author: sik@eia.udg.edu}
\usetheme{Autumn}\usecolorstyle[colorPalette=BrownBlueOrange]{Germany}
%\usetheme{Simple} \usecolorstyle[colorOne=udgColor]{Denmark} 

\input{defineBreastGTlabelColors.tex}

\begin{document}\maketitle
/home/sik/Work/escola/recerca/iwdm2014/paper/acronyms.tex
\acresetall

\graphicspath{{figures/paperFigures/}}


\begin{columns} 
  \column{0.6}
\block{Abstract}{
Texture is a powerful cue for describing structures that show a high degree of similarity in their image intensity patterns. 
This paper describes the use of \acf{sift}, both as low-level and high-level descriptors, applied to differentiate the tissues present in breast US images. 

For such a task, a subset of 16 images has been randomly selected from a larger dataset of 700 \ac{us} images acquired at the \emph{UDIAT Diagnostic Centre of Parc Taul\'{i}} in Sabadell (Catalunya), between 2010 and 2012.
This subset has been complemented with multi-label \ac{gt}, as illustrated in figure~\ref{fig:dataExample}.

Experimental results are provided showing the validity of the proposed approach for describing the tissues in breast US images.

}
  \column{0.4}
\block{}{
\begin{tikzfigure}[Dataset sample. From left to right: image sample, accompanying multi-label \ac{gt}, tissue label \ac{gt} color-coding.]
  \centering
  \documentclass[border=2pt]{standalone}
\usepackage{tikz}


\input{../../defineBreastGTlabelColors.tex}

\begin{document}
\begin{tikzpicture}

\node[anchor=south west,inner sep=0] (imgNode) at (0,0) {\includegraphics[width=.12\textwidth]{pngImgs/gt/000002.png}};

\node[anchor=south east,inner sep=0] at (-8pt,0) {\includegraphics[width=.12\textwidth]{pngImgs/000002.png}};

\definecolor{lungColor}{rgb}{0.2353, 0.6078, 0.8235}
\definecolor{chestWallColor}{rgb}{0.5294, 0.7843, 0.6078}
\definecolor{ribColor}{rgb}{0.0000, 0.4510, 0.1961}
\definecolor{pectoralColor}{rgb}{0.6510, 0.3490, 1.0000}
\definecolor{fibroGlandColor}{rgb}{1.0000, 1.0000, 0.0000}
\definecolor{fatColor}{rgb}{0.9804, 0.5882, 0.1176}
\definecolor{skinColor}{rgb}{0.9804, 0.7255, 0.7451}
\definecolor{unkTissueColor}{rgb}{0.6000, 0.3020, 0.2510}
\definecolor{bgColor}{rgb}{0.0000, 0.0000, 0.0000}
\definecolor{lesionColor}{rgb}{1.0000, 0.2510, 0.0000}
\definecolor{boundaryColor}{rgb}{0.8784, 0.8784, 0.7529}

\tikzset{nameSt/.style=
{anchor=north west,rectangle,
node distance=15pt,
minimum width=8pt,minimum height=8pt,
}}
%\begin{scriptsize}
%
\draw[] (imgNode.north east) +(8pt,0) node[nameSt, draw=chestWallColor, fill=chestWallColor,label=right:Chest wall] (cwName) {} ;
\draw[] node[nameSt, draw=ribColor, fill=ribColor,below=of cwName,label=right:Rib] (ribName) {} ;
\draw[] node[nameSt, draw=skinColor, fill=skinColor,below=of ribName,label=right:Skin layers] (skinName) {} ;
%\draw[] node[nameSt, draw=unkTissueColor, fill=unkTissueColor,below=of skinName,label=right:Chest wall] (unkTissueName) {} ;

\draw[] node[nameSt, draw=lesionColor, fill=lesionColor,below=of skinName,label=right:Lesion] (lesionName) {} ;
\draw[] node[nameSt, draw=boundaryColor, fill=boundaryColor,below=of lesionName,label=right:Boundary] (boundaryName) {} ;

\draw[] (imgNode.north east) +(8,0) node[nameSt,draw=lungColor, fill=lungColor,label=right:Air or lungs] (lungName) {};
\draw[] node[nameSt, draw=pectoralColor, fill=pectoralColor,below=of lungName,label=right:Pectoral muscle] (pectoralName) {} ;
\draw[] node[nameSt, draw=fibroGlandColor, fill=fibroGlandColor,below=of pectoralName,label=right:Fibro-glandular tissue] (fibroGlandName) {} ;
\draw[] node[nameSt, draw=fatColor, fill=fatColor,below=of fibroGlandName,label=right:Adipose tissue] (fatName) {} ;
\draw[] node[nameSt, draw=bgColor, fill=bgColor,below=of fatName,label=right:Background] (bgName) {} ;
%
%\end{scriptsize}

\end{tikzpicture}
\end{document}

  \label{fig:dataExample}
\end{tikzfigure}
}
\end{columns}
%\begin{keywords}
%breast cancer, ultrasound, texture, SIFT
%\end{keywords}

\block{\ac{sift} as a low-level descriptor, tested using \ac{map}}{
  In this experiment, it has been analyzed how separable are the tissue classes present in breast ultrasound images, when using low-level descriptors based on \ac{sift} to encode \ac{us} texture.
  Here a Bayesian framework has been assumed to perform the tissue discrimination and its results are presented both qualitatively (see fig.~\ref{fig:model}-\ref{fig:MAP}) and quantitatively (see fig.~\ref{fig:llConfMatrix}).
  
  All the pixel positions of all the images are used as a key-point for extracting a \ac{sift} descriptor and mapped in the $128D$ feature space of \ac{sift}.
  This \ac{sift} space is then projected into a $2D$ space to visually assess how the tissue classes are distributed in such space.
  From this projected space, models (see fig.~\ref{fig:model}) and priors (see fig.~\ref{fig:prior}) are extracted to infer the \ac{map} probability.
  Figure~\ref{fig:MAP} shows this \ac{map} probability illustrating how the tissue classes are separated based on the observed data.

In order to generate cross-validated quantitative results, the descriptors have been randomly sampled as follows: ($10.000\text{ samples} \times 10 \text{ classes}) \times 5 \text{ folds}$.
 At each round 4 folds have been used for training the %\ac{ml} term in eq.\,\ref{eq:bayes} ($P(\bar{x}_a|\omega)$) 
 models 
 and the remaining fold has been used for testing.
 Figure~\ref{fig:llConfMatrix} shows the corss-validated confusion matrix resulting from classifying breast tissues based on a Bayesian framework using either low-level \ac{sift} texture descriptors or intensity.
}

\begin{columns} 

\column{0.35} \block{}{
\begin{tikzfigure}[Distribution of the \acs{sift} descriptors for some classes in the \ac{gt}.]
\documentclass[border=2pt]{standalone}
\usepackage{tikz}


\input{../../defineBreastGTlabelColors.tex}

\begin{document}
 \begin{tikzpicture}
\tikzset{myNode/.style=
{anchor=north west,rectangle,
node distance=5pt,
minimum width=8pt,minimum height=8pt,
inner sep=0,
}}

\newcommand\mySize{6cm}

\node[myNode, label=below:\small Background] (bgNode) at (0,0) {\includegraphics[width=\mySize]{gtDistro/000.png}};
\node[myNode, right=of bgNode, label=below:\small tiny Air or lungs] (airNode) {\includegraphics[width=\mySize]{gtDistro/001.png}};
\node[myNode, right=of airNode, label=below:\small Chest wall] (cwNode) {\includegraphics[width=\mySize]{gtDistro/002.png}};

\node[myNode, right=of cwNode, label=below:\small Rib] (ribNode) {\includegraphics[width=\mySize]{gtDistro/003.png}};

\node[myNode, below=40pt of bgNode, label=below:\small Fibro-glandular] (fibNode) {\includegraphics[width=\mySize]{gtDistro/005.png}};

\node[myNode, right=of fibNode, label=below:\small Adipose tissue] (fatNode) {\includegraphics[width=\mySize]{gtDistro/006.png}};
\node[myNode, right=of fatNode, label=below:\small Skin layers] (skNode) {\includegraphics[width=\mySize]{gtDistro/007.png}};
\node[myNode, right=of skNode, label=below:\small Lesion] (lesionNode) {\includegraphics[width=\mySize]{gtDistro/128.png}};

\end{tikzpicture} 
\end{document}

  \label{fig:model}
\end{tikzfigure}
}

\column{0.15} \block{}{
\begin{tikzfigure}[\acs{sift} space. (a) Projected space colored according to \acs{gt} tissue labeling. (b) $P(\bar{x}_a)$. (c) $P(\omega)$.]
  \documentclass[border=2pt]{standalone}
\usepackage{tikz}
\usepackage{pgfplots,pgfplotstable}
\pgfplotsset{compat=1.4}
\usetikzlibrary{pgfplots.groupplots}

%\graphicspath{{../paperFigures/}}
\input{../../defineBreastGTlabelColors.tex}

\begin{document}

\pgfplotstableread[col sep=comma]{
Class, Occurrence, Pw
%-------------------------------------
Bg,     518738,	0.207
Lungs,  447153,	0.178
CWall,  134165,	0.054
Ribs,   22541, 	0.009
Pect,	  182468,	0.073
FGlan,  525238,	0.210
Fat,    413118,	0.165
Skin,   17732, 	0.007
Lsion,  163848,	0.065
Bound,  81082, 	0.032
}\occurrenceDataTable

\pgfplotscreateplotcyclelist{colorbrewer-RYB}{
{bgColor!50!black,fill=bgColor},
{lungColor!50!black,fill=lungColor},
{chestWallColor!50!black,fill=chestWallColor},
{ribColor!50!black,fill=ribColor},
{pectoralColor!50!black,fill=pectoralColor},
{fibroGlandColor!50!black,fill=fibroGlandColor},
{fatColor!50!black,fill=fatColor},
{skinColor!50!black,fill=skinColor},
{lesionColor!50!black,fill=lesionColor},
{boundaryColor!50!black,fill=boundaryColor},
}

\pgfplotsset{
    select row/.style={
        x filter/.code={\ifnum\coordindex=#1\else\def\pgfmathresult{}\fi}
    }
}

\newcommand\mySize{6cm} 

\begin{tikzpicture}
\begin{axis}[ 
              ybar=0pt,
              /pgf/bar shift=0pt,scale only axis,
              width=\mySize,
              ymin=0,
              xtick=\empty,
              tickpos=left,
              scaled y ticks=base 10:2,
              cycle list name=colorbrewer-RYB,
              bar width=14pt,
              ymajorgrids,
              font=\tiny,
              title=\small $P(\omega)$,
              name=pwAxis,
            ] 


\pgfplotsinvokeforeach  {0,...,9}{
  \addplot table [ x expr=\coordindex, select row=#1, y=Pw] {\occurrenceDataTable};
}
\end{axis}

\node at (pwAxis.south) 
      [ anchor=north,
        inner sep=0,
        yshift=-10pt,
%        draw,
        label=below:\small $P(\bar{x})$,
      ]{\includegraphics[height=\mySize]{siftOccurrences2}};
\end{tikzpicture}
\end{document}

  \label{fig:prior}
\end{tikzfigure}
} 

\column{0.15} \block{}{
\begin{tikzfigure}[ Qualitative evaluation of the \ac{map} labeling of the feature space.]
\documentclass[border=2pt]{standalone}
\usepackage{tikz}

\begin{document}
\begin{tikzpicture}
\node[anchor=south west,inner sep=0] (imgNode) at (0,0) {\includegraphics[trim = 91 230 90 230, clip,width=10cm]{labelingMAP.pdf}};

%\node[anchor=south east,inner sep=0] {\includegraphics[trim = 85 230 180 230, clip,width=.1\textwidth]{intensityLabelingMAP.pdf}};
\end{tikzpicture}
\end{document}

  \label{fig:MAP}
\end{tikzfigure}
}

\column{0.40} \block{}{
\begin{tikzfigure}[some caption]
\documentclass[border=2pt]{standalone}
\usepackage{tikz}
\usepackage{pgfplots}

\definecolor{bgColor}{rgb}{0.0000, 0.0000, 0.0000}
\definecolor{boundaryColor}{rgb}{0.8784, 0.8784, 0.7529}
\definecolor{chestWallColor}{rgb}{0.5294, 0.7843, 0.6078}
\definecolor{fatColor}{rgb}{0.9804, 0.5882, 0.1176}
\definecolor{fibroGlandColor}{rgb}{1.0000, 1.0000, 0.0000}
\definecolor{lesionColor}{rgb}{1.0000, 0.2510, 0.0000}
\definecolor{lungColor}{rgb}{0.2353, 0.6078, 0.8235}
\definecolor{pectoralColor}{rgb}{0.6510, 0.3490, 1.0000}
\definecolor{ribColor}{rgb}{0.0000, 0.4510, 0.1961}
\definecolor{skinColor}{rgb}{0.9804, 0.7255, 0.7451}
\definecolor{unkTissueColor}{rgb}{0.6000, 0.3020, 0.2510}

\begin{document}
\begin{tikzpicture}
\newcommand\confMatrixSize{7cm}
\newcommand\gtLabelDistance{0.3cm}

\tikzset{
every pin/.style={fill=yellow!50!white,rectangle,rounded corners=3pt,font=\tiny},
small dot/.style={fill=black,circle,scale=0.3}
}

\pgfplotsset{
	myStyleCommon/.append style={	width=\confMatrixSize,height=\confMatrixSize,
									scale only axis,
									xlabel={Estimated Class},
								    ylabel={Actual Class},
								    yticklabels={},
								    xticklabels={},
									xticklabel pos=right,
									xlabel near ticks,
									xtick={1.5,2.5,...,9.5},
									ytick={1.5,2.5,...,9.5},
									grid,
      }
}

\pgfplotsset{
	confMatrix/.append	style={	myStyleCommon,
								axis on top,
								colormap/jet,
								point meta min=0,
								point meta max=1,
								xmin=0.5,
								xmax=10.5,
								y dir=reverse,
								ymin=0.5,
								ymax=10.5,
							  }
}
\pgfplotsset{							  
	gtLabelsAxis/.append style={myStyleCommon,
%								axis lines=none,
								colormap={labelsGT}{color(0cm)=(bgColor);
					 								color(1cm)=(lungColor);	
					 								color(2cm)=(chestWallColor);
										 			color(3cm)=(ribColor);
										 			color(4cm)=(pectoralColor);
										 			color(5cm)=(fibroGlandColor); 
										 			color(6cm)=(fatColor); 
					 								color(7cm)=(skinColor);    
								 					color(8cm)=(lesionColor);
							 						color(9cm)=(boundaryColor); 
					 								color(10cm)=(unkTissueColor); 
					 								},
							    point meta min=0,
							    point meta max=10,
							    xmin=-0.5,xmax=9.5,
%							    height=0cm,
							  }				  
}
\tikzstyle{gtLabelsPlotStyle}=[scatter,only marks,mark=square*,mark size=5pt,domain=0:9,samples at={0,...,9}]


\begin{axis}[confMatrix,name=siftConfMatrix,clip = false,
every axis x label/.style= {at={(axis description cs:0.5,1.1)},draw=red,anchor=north},
every axis y label/.append style= {at={(axis description cs:-0.1,0.5)},draw=red,anchor=east},
yticklabel style={draw=red},
xticklabel style={draw=red},
]
\addplot graphics [xmin=0.5,xmax=10.5,ymin=0.5,ymax=10.5] {LowLevelSIFTConfusionMatrixMean.png};

\addplot[
scatter,
only marks,
mark size=3pt,
scatter src=explicit,
%point meta=explicit symbolic,
]
table[
y=actualClass,x=estimatedClass,meta=siftMean
 ]{lowLevelFeaturesConfusion.dat};
 
\node[small dot,pin=120:{xlabel anchor}]     at (axis description cs:0.5,1.1) {};           
\node[small dot,pin=-30:{ylabel anchor}]     at (axis description cs:-0.1,0.5) {};         

\end{axis}

\begin{axis}[gtLabelsAxis,
			 at=(siftConfMatrix.north),anchor=south,
			 yshift=\gtLabelDistance,
			 name=xGTLabels,
			 ]
			\addplot [gtLabelsPlotStyle,point meta=x] {0};
\end{axis}

 
\begin{axis}[gtLabelsAxis,%rotate=-90,yshift=-\gtLabelDistance,
			height=0cm,
			 name=yGTLabels,clip=false,
			 ]
			\addplot [gtLabelsPlotStyle,point meta=x] {0};   
%			\node[small dot,pin=125:{putaaaaaa}] at (axis description cs:0.5,0.5) {}; 
%			\node[small dot,pin=120:{$(0,0)$}]     at (axis description cs:0,0) {};           
%\node[small dot,pin=-30:{$(1,1)$}]     at (axis description cs:1,1) {};         
%\node[small dot,pin=-90:{$(1.03,0.5)$}]at (axis description cs:1.03,0.5) {}; 

			
\end{axis}


\end{tikzpicture}

\begin{tikzpicture}

\end{tikzpicture}



\end{document}
  \label{fig:llConfMatrix}
\end{tikzfigure}
} 

\end{columns}

\block{\ac{sift} as a high-level descriptor using \acf{bof}, tested using \ac{rbf}-\ac{svm} classifier}{
The high-level texture descriptor is build as a \ac{bof} of \ac{sift} descriptors. 
}

\begin{columns}
\column{0.60} \block{}{
\begin{tikzfigure}[ \acs{sift}-\acs{bof} descriptors qualitative analysis. (Left) image example. (Right) Dictionary representation colored using the location of the keypoint location in fig.\,\ref{fig:siftMapping}a space. (1-8) Occurrence of the dictionary's key-points associated to each region highlighted in the original image.]
\documentclass[border=2pt]{standalone}
\usepackage{tikz}


\input{../../defineBreastGTlabelColors.tex}

\begin{document}
\begin{tikzpicture}
\node[anchor=south west,inner sep=0] (imgNode) at (0,0) {\includegraphics[trim = 91 300 100 242, clip,width=.2\textwidth]{appearance1.pdf}};

\tikzset{nameSt/.style=
{anchor=north west,rectangle,
node distance=10pt,
minimum width=8pt,minimum height=8pt,
inner sep=0,
}}
\begin{tiny}

\draw[] (imgNode.north east) +(5pt,0) node[nameSt, label=below:Codebook] (dictionary) {\includegraphics[width=.10\textwidth,]{dictionary/01.png}} ;
\draw[] (dictionary.north east) +(5pt,0) node[nameSt, label=below:(1)] (BoFi) {\includegraphics[width=.10\textwidth,]{dictionary/01Signatures/127.png}} ;
\draw[] (BoFi.north east) +(5pt,0) node[nameSt, label=below:(2)] (BoFii) {\includegraphics[width=.02\textwidth,]{dictionary/01Signatures/037.png}} ;

\node[nameSt, below=of dictionary, label=below:(3)] (BoFiii) {\includegraphics[width=.02\textwidth,]{dictionary/01Signatures/054.png}} ;
\node[nameSt, below=of BoFi, label=below:(4)] (BoFiv) {\includegraphics[width=.02\textwidth,]{dictionary/01Signatures/118.png}} ;
\node[nameSt, below=of BoFii, label=below:(5)] (BoFv) {\includegraphics[width=.02\textwidth,]{dictionary/01Signatures/160.png}} ;

\node[nameSt, below=of BoFiii, label=below:(6)] (BoFvi) {\includegraphics[width=.02\textwidth,]{dictionary/01Signatures/061.png}} ;
\node[nameSt, below=of BoFiv, label=below:(7)] (BoFvii) {\includegraphics[width=.02\textwidth,]{dictionary/01Signatures/077.png}} ;
\node[nameSt, below=of BoFv, label=below:(8)] (BoFviii) {\includegraphics[width=.02\textwidth,]{dictionary/01Signatures/043.png}} ;

\end{tiny}

\end{tikzpicture}
\end{document}
 
\end{tikzfigure}
}
\column{0.40} \block{}{
\begin{tikzfigure}[some caption]
\documentclass[border=2pt]{standalone}
\usepackage{tikz}
\usepackage{pgfplots}

\definecolor{bgColor}{rgb}{0.0000, 0.0000, 0.0000}
\definecolor{boundaryColor}{rgb}{0.8784, 0.8784, 0.7529}
\definecolor{chestWallColor}{rgb}{0.5294, 0.7843, 0.6078}
\definecolor{fatColor}{rgb}{0.9804, 0.5882, 0.1176}
\definecolor{fibroGlandColor}{rgb}{1.0000, 1.0000, 0.0000}
\definecolor{lesionColor}{rgb}{1.0000, 0.2510, 0.0000}
\definecolor{lungColor}{rgb}{0.2353, 0.6078, 0.8235}
\definecolor{pectoralColor}{rgb}{0.6510, 0.3490, 1.0000}
\definecolor{ribColor}{rgb}{0.0000, 0.4510, 0.1961}
\definecolor{skinColor}{rgb}{0.9804, 0.7255, 0.7451}
\definecolor{unkTissueColor}{rgb}{0.6000, 0.3020, 0.2510}

\begin{document}
\begin{tikzpicture}
\newcommand\confMatrixSize{7cm}
\newcommand\gtLabelDistance{0.3cm}

\tikzset{
every pin/.style={fill=yellow!50!white,rectangle,rounded corners=3pt,font=\tiny},
small dot/.style={fill=black,circle,scale=0.3}
}

\pgfplotsset{
	myStyleCommon/.append style={	width=\confMatrixSize,height=\confMatrixSize,
									scale only axis,
									xlabel={Estimated Class},
								    ylabel={Actual Class},
								    yticklabels={},
								    xticklabels={},
									xticklabel pos=right,
									xlabel near ticks,
									xtick={1.5,2.5,...,9.5},
									ytick={1.5,2.5,...,9.5},
									grid,
      }
}

\pgfplotsset{
	confMatrix/.append	style={	myStyleCommon,
								axis on top,
								colormap/jet,
								point meta min=0,
								point meta max=1,
								xmin=0.5,
								xmax=10.5,
								y dir=reverse,
								ymin=0.5,
								ymax=10.5,
							  }
}
\pgfplotsset{							  
	gtLabelsAxis/.append style={myStyleCommon,
%								axis lines=none,
								colormap={labelsGT}{color(0cm)=(bgColor);
					 								color(1cm)=(lungColor);	
					 								color(2cm)=(chestWallColor);
										 			color(3cm)=(ribColor);
										 			color(4cm)=(pectoralColor);
										 			color(5cm)=(fibroGlandColor); 
										 			color(6cm)=(fatColor); 
					 								color(7cm)=(skinColor);    
								 					color(8cm)=(lesionColor);
							 						color(9cm)=(boundaryColor); 
					 								color(10cm)=(unkTissueColor); 
					 								},
							    point meta min=0,
							    point meta max=10,
							    xmin=-0.5,xmax=9.5,
%							    height=0cm,
							  }				  
}
\tikzstyle{gtLabelsPlotStyle}=[scatter,only marks,mark=square*,mark size=5pt,domain=0:9,samples at={0,...,9}]


\begin{axis}[confMatrix,name=siftConfMatrix,clip = false,
every axis x label/.style= {at={(axis description cs:0.5,1.1)},draw=red,anchor=north},
every axis y label/.append style= {at={(axis description cs:-0.1,0.5)},draw=red,anchor=east},
yticklabel style={draw=red},
xticklabel style={draw=red},
]
\addplot graphics [xmin=0.5,xmax=10.5,ymin=0.5,ymax=10.5] {LowLevelSIFTConfusionMatrixMean.png};

\addplot[
scatter,
only marks,
mark size=3pt,
scatter src=explicit,
%point meta=explicit symbolic,
]
table[
y=actualClass,x=estimatedClass,meta=siftMean
 ]{lowLevelFeaturesConfusion.dat};
 
\node[small dot,pin=120:{xlabel anchor}]     at (axis description cs:0.5,1.1) {};           
\node[small dot,pin=-30:{ylabel anchor}]     at (axis description cs:-0.1,0.5) {};         

\end{axis}

\begin{axis}[gtLabelsAxis,
			 at=(siftConfMatrix.north),anchor=south,
			 yshift=\gtLabelDistance,
			 name=xGTLabels,
			 ]
			\addplot [gtLabelsPlotStyle,point meta=x] {0};
\end{axis}

 
\begin{axis}[gtLabelsAxis,%rotate=-90,yshift=-\gtLabelDistance,
			height=0cm,
			 name=yGTLabels,clip=false,
			 ]
			\addplot [gtLabelsPlotStyle,point meta=x] {0};   
%			\node[small dot,pin=125:{putaaaaaa}] at (axis description cs:0.5,0.5) {}; 
%			\node[small dot,pin=120:{$(0,0)$}]     at (axis description cs:0,0) {};           
%\node[small dot,pin=-30:{$(1,1)$}]     at (axis description cs:1,1) {};         
%\node[small dot,pin=-90:{$(1.03,0.5)$}]at (axis description cs:1.03,0.5) {}; 

			
\end{axis}


\end{tikzpicture}

\begin{tikzpicture}

\end{tikzpicture}



\end{document}
\end{tikzfigure}
} 
\end{columns}


%\block[titleoffsety=-1cm,bodyoffsety=-1cm]{Sample document}{This poster...}
\block{Conclusion}{
The present study was designed to explore the usage of \ac{sift} feature space as a texture for characterizing the different tissues present in a breast \ac{us} image. 
The usage of \ac{sift} either as a low-level or high-level texture descriptor has been evaluated in comparison to intensity features, which are the features most commonly used.
The fact that \ac{sift} and intensity descriptors produce similar results, encourages further studies on using \ac{sift} texture descriptors characterizing breast tissues in \ac{us} images.

}
%\note[targetoffsetx=24cm, targetoffsety=-9cm,radius=8cm,width=.75\textwidth,innersep=.4cm]{You can...}
\note[targetoffsetx=0cm,targetoffsety=-9cm,radius=8cm,width=.5\textwidth,innersep=.4cm]{You can...}
\end{document}
