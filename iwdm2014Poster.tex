\documentclass[25pt, a0paper, portrait, margin=0mm, innermargin=15mm, blockverticalspace=15mm, colspace=15mm, subcolspace=8mm]{tikzposter}

\usepackage{amsmath}
\usepackage{epsf,graphicx,subfig}
\setcounter{tocdepth}{3}
\usepackage{xcolor}
\usepackage{lineno}
\usepackage[nolist]{acronym}
\usepackage{epsf,graphicx,subfig}
\usepackage{amssymb,amsmath}
\usepackage{tikz}
\usetikzlibrary{positioning}
\usepackage{todonotes}
\usepackage{standalone}
\usepackage{scalefnt}
\usepackage{url}

\definecolor{udgColor}{RGB}{82,119,213}


\title{\acs{sift} texture description for understanding breast ultrasound images}
%\author{Joan Massich\inst{1}\inst{2}\thanks{This work was partially supported by the Spanish Science and Innovation grant nb. TIN2012-37171-C02-01 and TTIN2012-37171-C02-02 and the Regional Council of Burgundy.} \and Fabrice Meriaudeau \inst{2} \and Melcior Sent{\'i}s\inst{3} \and Sergi Ganau\inst{3} \and Elsa~P{\'e}rez\inst{4} 
%\and Domenec Puig\inst{5} \and Robert  Mart{\'i} \inst{1} \and  Arnau Oliver\inst{1}\and Joan Mart{\'i} \inst{1}}
%\institute{Computer Vision and Robotics Group, University of Girona, Spain. \email{jmassich@atc.udg.edu} \and
%Laboratoire Le2i-UMR CNRS, University of Burgundy,  Le Creusot, France.
% \and Department of
%Breast and Gynecological Radiology,  UDIAT-Diagnostic Center, Parc
%Taul{\'i} Corporation, Sabadell, Spain.
% \and 
%Department of Radiology, Hospital Josep Trueta of Girona, Spain.
%\and
%Department of Computer Engineering and Mathematics, University Rovira i Virgili, Tarragona, Spain.}

\author{Joan Massich, Fabrice Meriaudeau, Melcior Sent{\'i}s, Sergi Ganau, Elsa~P{\'e}rez, Domenec Puig, Robert  Mart{\'i}, Arnau Oliver and Joan Mart{\'i}}

%\institute{contact author: sik@eia.udg.edu}
%\usetheme{Autumn}\usecolorstyle[colorPalette=BrownBlueOrange]{Germany}
\usetheme{Simple}
%\usecolorstyle[colorOne=udgColor]{Russia} 
\usecolorstyle[colorOne=udgColor]{Denmark} 

\input{defineBreastGTlabelColors.tex}

\begin{document}\maketitle
/home/sik/Work/escola/recerca/iwdm2014/paper/acronyms.tex

\graphicspath{{figures/paperFigures/}}
\acresetall

\block{Abstract}{
Texture is a powerful cue for describing structures that show a high degree of similarity in their image intensity patterns. This paper describes the use of \acf{sift}, both as low-level and high-level descriptors, applied to differentiate the tissues present in breast US images. For the low-level texture descriptors case, \ac{sift} descriptors are extracted from a regular grid. The high-level texture descriptor is build as a \ac{bof} of \ac{sift} descriptors. 
Experimental results are provided showing the validity of the proposed approach for describing the tissues in breast US images.
}
%\begin{keywords}
%breast cancer, ultrasound, texture, SIFT
%\end{keywords}

\begin{columns} \column{0.45}

\block{Problem definition}{
 \begin{tikzfigure}[Dataset sample. From left to right: image sample, accompanying multi-label \ac{gt}, tissue label \ac{gt} color-coding.]
      \centering
	\documentclass[border=2pt]{standalone}
\usepackage{tikz}


\input{../../defineBreastGTlabelColors.tex}

\begin{document}
\begin{tikzpicture}

\node[anchor=south west,inner sep=0] (imgNode) at (0,0) {\includegraphics[width=.12\textwidth]{pngImgs/gt/000002.png}};

\node[anchor=south east,inner sep=0] at (-8pt,0) {\includegraphics[width=.12\textwidth]{pngImgs/000002.png}};

\definecolor{lungColor}{rgb}{0.2353, 0.6078, 0.8235}
\definecolor{chestWallColor}{rgb}{0.5294, 0.7843, 0.6078}
\definecolor{ribColor}{rgb}{0.0000, 0.4510, 0.1961}
\definecolor{pectoralColor}{rgb}{0.6510, 0.3490, 1.0000}
\definecolor{fibroGlandColor}{rgb}{1.0000, 1.0000, 0.0000}
\definecolor{fatColor}{rgb}{0.9804, 0.5882, 0.1176}
\definecolor{skinColor}{rgb}{0.9804, 0.7255, 0.7451}
\definecolor{unkTissueColor}{rgb}{0.6000, 0.3020, 0.2510}
\definecolor{bgColor}{rgb}{0.0000, 0.0000, 0.0000}
\definecolor{lesionColor}{rgb}{1.0000, 0.2510, 0.0000}
\definecolor{boundaryColor}{rgb}{0.8784, 0.8784, 0.7529}

\tikzset{nameSt/.style=
{anchor=north west,rectangle,
node distance=15pt,
minimum width=8pt,minimum height=8pt,
}}
%\begin{scriptsize}
%
\draw[] (imgNode.north east) +(8pt,0) node[nameSt, draw=chestWallColor, fill=chestWallColor,label=right:Chest wall] (cwName) {} ;
\draw[] node[nameSt, draw=ribColor, fill=ribColor,below=of cwName,label=right:Rib] (ribName) {} ;
\draw[] node[nameSt, draw=skinColor, fill=skinColor,below=of ribName,label=right:Skin layers] (skinName) {} ;
%\draw[] node[nameSt, draw=unkTissueColor, fill=unkTissueColor,below=of skinName,label=right:Chest wall] (unkTissueName) {} ;

\draw[] node[nameSt, draw=lesionColor, fill=lesionColor,below=of skinName,label=right:Lesion] (lesionName) {} ;
\draw[] node[nameSt, draw=boundaryColor, fill=boundaryColor,below=of lesionName,label=right:Boundary] (boundaryName) {} ;

\draw[] (imgNode.north east) +(8,0) node[nameSt,draw=lungColor, fill=lungColor,label=right:Air or lungs] (lungName) {};
\draw[] node[nameSt, draw=pectoralColor, fill=pectoralColor,below=of lungName,label=right:Pectoral muscle] (pectoralName) {} ;
\draw[] node[nameSt, draw=fibroGlandColor, fill=fibroGlandColor,below=of pectoralName,label=right:Fibro-glandular tissue] (fibroGlandName) {} ;
\draw[] node[nameSt, draw=fatColor, fill=fatColor,below=of fibroGlandName,label=right:Adipose tissue] (fatName) {} ;
\draw[] node[nameSt, draw=bgColor, fill=bgColor,below=of fatName,label=right:Background] (bgName) {} ;
%
%\end{scriptsize}

\end{tikzpicture}
\end{document}

\end{tikzfigure}
}


\block{xxxxxxxxxx}{
\begin{tikzfigure}[\acs{sift} space. (a) Projected space colored according to \acs{gt} tissue labeling. (b) $P(\bar{x}_a)$. (c) $P(\omega)$.]
\documentclass[border=2pt]{standalone}
\usepackage{tikz}
\usepackage{pgfplots,pgfplotstable}
\pgfplotsset{compat=1.4}
\usetikzlibrary{pgfplots.groupplots}

%\graphicspath{{../paperFigures/}}
\input{../../defineBreastGTlabelColors.tex}

\begin{document}

\pgfplotstableread[col sep=comma]{
Class, Occurrence, Pw
%-------------------------------------
Bg,     518738,	0.207
Lungs,  447153,	0.178
CWall,  134165,	0.054
Ribs,   22541, 	0.009
Pect,	  182468,	0.073
FGlan,  525238,	0.210
Fat,    413118,	0.165
Skin,   17732, 	0.007
Lsion,  163848,	0.065
Bound,  81082, 	0.032
}\occurrenceDataTable

\pgfplotscreateplotcyclelist{colorbrewer-RYB}{
{bgColor!50!black,fill=bgColor},
{lungColor!50!black,fill=lungColor},
{chestWallColor!50!black,fill=chestWallColor},
{ribColor!50!black,fill=ribColor},
{pectoralColor!50!black,fill=pectoralColor},
{fibroGlandColor!50!black,fill=fibroGlandColor},
{fatColor!50!black,fill=fatColor},
{skinColor!50!black,fill=skinColor},
{lesionColor!50!black,fill=lesionColor},
{boundaryColor!50!black,fill=boundaryColor},
}

\pgfplotsset{
    select row/.style={
        x filter/.code={\ifnum\coordindex=#1\else\def\pgfmathresult{}\fi}
    }
}

\newcommand\mySize{6cm} 

\begin{tikzpicture}
\begin{axis}[ 
              ybar=0pt,
              /pgf/bar shift=0pt,scale only axis,
              width=\mySize,
              ymin=0,
              xtick=\empty,
              tickpos=left,
              scaled y ticks=base 10:2,
              cycle list name=colorbrewer-RYB,
              bar width=14pt,
              ymajorgrids,
              font=\tiny,
              title=\small $P(\omega)$,
              name=pwAxis,
            ] 


\pgfplotsinvokeforeach  {0,...,9}{
  \addplot table [ x expr=\coordindex, select row=#1, y=Pw] {\occurrenceDataTable};
}
\end{axis}

\node at (pwAxis.south) 
      [ anchor=north,
        inner sep=0,
        yshift=-10pt,
%        draw,
        label=below:\small $P(\bar{x})$,
      ]{\includegraphics[height=\mySize]{siftOccurrences2}};
\end{tikzpicture}
\end{document}

\end{tikzfigure}
} 




\column{0.45} \block{xxxxxxx}{
\begin{tikzfigure}[Distribution of the \acs{sift} descriptors for some classes in the \ac{gt}.]
\documentclass[border=2pt]{standalone}
\usepackage{tikz}


\input{../../defineBreastGTlabelColors.tex}

\begin{document}
 \begin{tikzpicture}
\tikzset{myNode/.style=
{anchor=north west,rectangle,
node distance=5pt,
minimum width=8pt,minimum height=8pt,
inner sep=0,
}}

\newcommand\mySize{6cm}

\node[myNode, label=below:\small Background] (bgNode) at (0,0) {\includegraphics[width=\mySize]{gtDistro/000.png}};
\node[myNode, right=of bgNode, label=below:\small tiny Air or lungs] (airNode) {\includegraphics[width=\mySize]{gtDistro/001.png}};
\node[myNode, right=of airNode, label=below:\small Chest wall] (cwNode) {\includegraphics[width=\mySize]{gtDistro/002.png}};

\node[myNode, right=of cwNode, label=below:\small Rib] (ribNode) {\includegraphics[width=\mySize]{gtDistro/003.png}};

\node[myNode, below=40pt of bgNode, label=below:\small Fibro-glandular] (fibNode) {\includegraphics[width=\mySize]{gtDistro/005.png}};

\node[myNode, right=of fibNode, label=below:\small Adipose tissue] (fatNode) {\includegraphics[width=\mySize]{gtDistro/006.png}};
\node[myNode, right=of fatNode, label=below:\small Skin layers] (skNode) {\includegraphics[width=\mySize]{gtDistro/007.png}};
\node[myNode, right=of skNode, label=below:\small Lesion] (lesionNode) {\includegraphics[width=\mySize]{gtDistro/128.png}};

\end{tikzpicture} 
\end{document}

\end{tikzfigure}
}


\block{xxx}{
\begin{tikzfigure}[ Qualitative evaluation of the \ac{map} labeling of the feature space.]
\documentclass[border=2pt]{standalone}
\usepackage{tikz}

\begin{document}
\begin{tikzpicture}
\node[anchor=south west,inner sep=0] (imgNode) at (0,0) {\includegraphics[trim = 91 230 90 230, clip,width=10cm]{labelingMAP.pdf}};

%\node[anchor=south east,inner sep=0] {\includegraphics[trim = 85 230 180 230, clip,width=.1\textwidth]{intensityLabelingMAP.pdf}};
\end{tikzpicture}
\end{document}

\end{tikzfigure}
}



\block{xxx}{
\begin{tikzfigure}[ \acs{sift}-\acs{bof} descriptors qualitative analysis. (Left) image example. (Right) Dictionary representation colored using the location of the keypoint location in fig.\,\ref{fig:siftMapping}a space. (1-8) Occurrence of the dictionary's key-points associated to each region highlighted in the original image.]
\documentclass[border=2pt]{standalone}
\usepackage{tikz}


\input{../../defineBreastGTlabelColors.tex}

\begin{document}
\begin{tikzpicture}
\node[anchor=south west,inner sep=0] (imgNode) at (0,0) {\includegraphics[trim = 91 300 100 242, clip,width=.2\textwidth]{appearance1.pdf}};

\tikzset{nameSt/.style=
{anchor=north west,rectangle,
node distance=10pt,
minimum width=8pt,minimum height=8pt,
inner sep=0,
}}
\begin{tiny}

\draw[] (imgNode.north east) +(5pt,0) node[nameSt, label=below:Codebook] (dictionary) {\includegraphics[width=.10\textwidth,]{dictionary/01.png}} ;
\draw[] (dictionary.north east) +(5pt,0) node[nameSt, label=below:(1)] (BoFi) {\includegraphics[width=.10\textwidth,]{dictionary/01Signatures/127.png}} ;
\draw[] (BoFi.north east) +(5pt,0) node[nameSt, label=below:(2)] (BoFii) {\includegraphics[width=.02\textwidth,]{dictionary/01Signatures/037.png}} ;

\node[nameSt, below=of dictionary, label=below:(3)] (BoFiii) {\includegraphics[width=.02\textwidth,]{dictionary/01Signatures/054.png}} ;
\node[nameSt, below=of BoFi, label=below:(4)] (BoFiv) {\includegraphics[width=.02\textwidth,]{dictionary/01Signatures/118.png}} ;
\node[nameSt, below=of BoFii, label=below:(5)] (BoFv) {\includegraphics[width=.02\textwidth,]{dictionary/01Signatures/160.png}} ;

\node[nameSt, below=of BoFiii, label=below:(6)] (BoFvi) {\includegraphics[width=.02\textwidth,]{dictionary/01Signatures/061.png}} ;
\node[nameSt, below=of BoFiv, label=below:(7)] (BoFvii) {\includegraphics[width=.02\textwidth,]{dictionary/01Signatures/077.png}} ;
\node[nameSt, below=of BoFv, label=below:(8)] (BoFviii) {\includegraphics[width=.02\textwidth,]{dictionary/01Signatures/043.png}} ;

\end{tiny}

\end{tikzpicture}
\end{document}
 
\end{tikzfigure}
}



\end{columns}
\block[titleoffsety=-1cm,bodyoffsety=-1cm]{Sample document}{This poster...}
%\note[targetoffsetx=24cm, targetoffsety=-9cm,radius=8cm,width=.75\textwidth,innersep=.4cm]{You can...}
\note[targetoffsetx=0cm,targetoffsety=-9cm,radius=8cm,width=.5\textwidth,innersep=.4cm]{You can...}
\end{document}
